

\documentclass[english]{article}
\usepackage{fullpage}
\usepackage[T1]{fontenc}
\usepackage[latin9]{inputenc}
\setlength{\parskip}{\medskipamount}
\setlength{\parindent}{0pt}



\usepackage{amsmath}
\usepackage{amssymb}
\usepackage{amsthm}
\usepackage{amsfonts}
\usepackage{graphicx}
\usepackage{mathtools}
\usepackage{hyperref}

%\usepackage[g]{esvect}

\DeclarePairedDelimiter{\abs}{\lvert}{\rvert}
\DeclarePairedDelimiter{\ceil}{\lceil}{\rceil}
\DeclarePairedDelimiter{\floor}{\lfloor}{\rfloor}
\DeclarePairedDelimiter{\avect}{\langle}{\rangle}
\DeclarePairedDelimiter{\aset}{\allowbreak\lbrace}{\rbrace}

%\newcommand{\true}{\mathit{true}}
%\newcommand{\fals}{\mathit{false}}
\newcommand{\true}{\mbox{T}}
\newcommand{\fals}{\mbox{F}}
\newcommand{\Nat}{\mathbb{N}}
\newcommand{\Int}{\mathbb{Z}}
\newcommand{\defeq}{:=}
\newcommand{\transpose}{^ \top }
\newcommand{\queseq}{\overset{?}{=}}
%\newcommand{\transpose}{^ \intercal }
\newcommand{\pipe}{\,|\,}
\newcommand{\vbl}[1]{\ensuremath{\mathit{#1}}}
\newcommand{\detop}[1]{\det(#1)}
%\newcommand{\vect}[1]{\boldsymbol{\vbl{#1}}}
\newcommand{\vect}[1]{{\overrightarrow{#1}}}
\newcommand{\slfrac}[2]{\left.#1\middle/#2\right.}
%\newcommand{\vect}[1]{{\vv{#1}}}
\newcommand{\modop}[1]{\mbox{ mod }#1}
\newcommand{\mathsc}[1]{\text{\normalfont\scshape#1}}

\newcounter{exercisecnt}
\def\theexercisecnt{\arabic{exercisecnt}}
\newenvironment{exercise}
{\refstepcounter{exercisecnt}
 {\bf Exercise \theexercisecnt.}}
{}

\newcounter{exercisepartcnt}[exercisecnt]
\def\theexercisepartcnt{\theexercisecnt.\alph{exercisepartcnt}}
\newenvironment{exercisepart}
{\refstepcounter{exercisepartcnt}
 {\bf Exercise \theexercisepartcnt.}}
{}



\makeatletter
\@ifundefined{ifusesection}{%
 \newif\ifusesection %
 \usesectiontrue %
}{}
\makeatother

\ifusesection
 \newtheorem{theorem}{Theorem}[section]
\else
 \newtheorem{theorem}{Theorem}
\fi

\newtheorem{lemma}[theorem]{Lemma}
\newtheorem{corollary}[theorem]{Corollary}
\newtheorem{definition}[theorem]{Definition}
\newtheorem{example}[theorem]{Example}

\renewcommand{\labelenumii}{\arabic{enumi}.\arabic{enumii}.}
\renewcommand{\labelenumiii}{\arabic{enumi}.\arabic{enumii}.\arabic{enumiii}.}
\renewcommand{\labelenumiv}{\arabic{enumi}.\arabic{enumii}.\arabic{enumiii}.\arabic{enumiv}.}



\usepackage{tikz}
\usetikzlibrary{arrows}

\begin{document}
\title{
 CS5811 Not Project Proposal:\\
 Search for Self-Stabilizing Protocols
}

\author{~Brandon~Crowley,~Alex~Klinkhamer, and~Mandy~Wang}
\maketitle



Comments on your proposal document:

{\it Is a protocol defined in a standard way? Show the protocol explicitly in the example you provided.}

YES

{\it Formulate the search problem: states, actions, goals.
\begin{itemize}
\item Show how BDDs represent states.
\item Show the input file format
\end{itemize}
}

{\it A SAT solver is a special case of a CSP solver. How will your methods be different from the SAT solver?}

It blows up in size when given to a SAT solver.

{\it How can the manual step in 1 be automated?}

TODO

{\it Aim to have a working system for a very simple example even if heuristics do not perform well.}

Sounds reasonable (no answer)

{\it Provide time estimates and tasks. Identify tasks for project group members.}
Our current to-do list consists primarily of constructing a detailed list of activites required to finish this project, with a rough timeline 
and initial task assignments for each member, all of which will be taken care of early in the next week, and then we will begin with whichever tasks top
the new list.  After that, our broader to-do list consists of the items laid out in section 2 of the proposal, less items already completed:
cycle checking, weak stabilization checking, the backtracking algorithm, implementing the most-constrained variable and AC-3 heuristics, and performing
tests to measure the effectiveness of our implementation.

\section{Examples}
Some examples of stabilizing protocols follow.
We plan to use these (and more) in our experimentation.

\subsection{Dijkstra's Token Ring}
Dijkstra introduced three token ring protocols when he introduced the concept of self-stabilization \cite{dij}.

\subsection{Maximal Matching}

\subsection{3-Coloring on a Ring}

\bibliographystyle{plain}
\bibliography{bibliography}

\end{document}

