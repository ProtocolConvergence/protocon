

\documentclass[english]{article}
\usepackage{fullpage}
\usepackage[T1]{fontenc}
\usepackage[latin9]{inputenc}
\setlength{\parskip}{\medskipamount}
\setlength{\parindent}{0pt}



\usepackage{amsmath}
\usepackage{amssymb}
\usepackage{amsthm}
\usepackage{amsfonts}
\usepackage{graphicx}
\usepackage{mathtools}
\usepackage{hyperref}

%\usepackage[g]{esvect}

\DeclarePairedDelimiter{\abs}{\lvert}{\rvert}
\DeclarePairedDelimiter{\ceil}{\lceil}{\rceil}
\DeclarePairedDelimiter{\floor}{\lfloor}{\rfloor}
\DeclarePairedDelimiter{\avect}{\langle}{\rangle}
\DeclarePairedDelimiter{\aset}{\allowbreak\lbrace}{\rbrace}

%\newcommand{\true}{\mathit{true}}
%\newcommand{\fals}{\mathit{false}}
\newcommand{\true}{\mbox{T}}
\newcommand{\fals}{\mbox{F}}
\newcommand{\Nat}{\mathbb{N}}
\newcommand{\Int}{\mathbb{Z}}
\newcommand{\defeq}{:=}
\newcommand{\transpose}{^ \top }
\newcommand{\queseq}{\overset{?}{=}}
%\newcommand{\transpose}{^ \intercal }
\newcommand{\pipe}{\,|\,}
\newcommand{\vbl}[1]{\ensuremath{\mathit{#1}}}
\newcommand{\detop}[1]{\det(#1)}
%\newcommand{\vect}[1]{\boldsymbol{\vbl{#1}}}
\newcommand{\vect}[1]{{\overrightarrow{#1}}}
\newcommand{\slfrac}[2]{\left.#1\middle/#2\right.}
%\newcommand{\vect}[1]{{\vv{#1}}}
\newcommand{\modop}[1]{\mbox{ mod }#1}
\newcommand{\mathsc}[1]{\text{\normalfont\scshape#1}}

\newcounter{exercisecnt}
\def\theexercisecnt{\arabic{exercisecnt}}
\newenvironment{exercise}
{\refstepcounter{exercisecnt}
 {\bf Exercise \theexercisecnt.}}
{}

\newcounter{exercisepartcnt}[exercisecnt]
\def\theexercisepartcnt{\theexercisecnt.\alph{exercisepartcnt}}
\newenvironment{exercisepart}
{\refstepcounter{exercisepartcnt}
 {\bf Exercise \theexercisepartcnt.}}
{}



\makeatletter
\@ifundefined{ifusesection}{%
 \newif\ifusesection %
 \usesectiontrue %
}{}
\makeatother

\ifusesection
 \newtheorem{theorem}{Theorem}[section]
\else
 \newtheorem{theorem}{Theorem}
\fi

\newtheorem{lemma}[theorem]{Lemma}
\newtheorem{corollary}[theorem]{Corollary}
\newtheorem{definition}[theorem]{Definition}
\newtheorem{example}[theorem]{Example}

\renewcommand{\labelenumii}{\arabic{enumi}.\arabic{enumii}.}
\renewcommand{\labelenumiii}{\arabic{enumi}.\arabic{enumii}.\arabic{enumiii}.}
\renewcommand{\labelenumiv}{\arabic{enumi}.\arabic{enumii}.\arabic{enumiii}.\arabic{enumiv}.}



\begin{document}
\title{Search Algorithm}
\author{Alex Klinkhamer}
\maketitle

\section{Depth-First Search}
I use a depth-first search when not using a SAT solver.

\begin{enumerate}
 \item Prune all actions which break closure.
 \item For each remaining action,
  \begin{enumerate}
   \item Add the action.
   \item If a livelock is created, remove from list of remaining actions.
   \item If all deadlocks are resolved, Return Yes.
   \item Note how many deadlocks the action resolves.
   \item Remove the action.
  \end{enumerate}
 \item Order actions by number of resolved deadlocks.
 \item If remaining actions do not add weak convergence, Return Nil.
 \item For each remaining action,
  \begin{enumerate}
   \item Add the action and recurse (to try remaining actions).
   \item If solution found, Return Yes.
   \item Remove the action.
   \item If there exist deadlock states which can only be resolved by a single action,
    \begin{enumerate}
     \item Add all required actions.
     \item Recurse (to try remaining actions).
     \item Return the result of the function call.
    \end{enumerate}
  \end{enumerate}
\end{enumerate}

\section{Future}
\begin{enumerate}
 \item Allow for weak fairness.
 \item If any illegitimate state is locally checkable, one of the local state variables must be changed, at least to something not locally-checkable.
  This is like the outermost ring of the "best ranking" algorithm.
\end{enumerate}

\subsection{Permutations}
Let vector $\vect{d}$ specify the domain of variables.
That is, $\vect{e}$ is a valuation of $\vect{d}$ iff $\forall i: \vect{e}[i]\in\Nat_{\vect{d}[i]}$.
Any valuation of $\vect{d}$ can be mapped uniquely to an integer from $0$ to $\prod_{i=0}^{\mathit{Size}(\vect{d})-1} \vect{d}[i]-1$.
That is, $\vect{e}[i]$ represents the $i$th digit of a variable-base number.
The $i$th digit is of base $\vect{d}[i]$.
Addition modulo $\vect{d}$ can be implemented with carry.

Looping through all valuations of $\vect{d}$ is equivalent to looping through all numbers representable by valuations of $\vect{d}$.
How does one do that?
We could of course start at $0$ and continue counting up by $1$ until wrapping around, giving the sequence $0,1,2,\dots,\vect{d}-1,0$.
Really though, we could start at a number $\vect{c}$ which is coprime with $\vect{d}$ and continue adding $\vect{c}$ to get the sequence $\vect{c},2\vect{c},3\vect{c},\dots,(\vect{d}-1)\vect{c},\vect{c}$, all modulo $\vect{d}$.
This can be achieved by repetitive addition of $\vect{c}$ instead of multiplication.

Use a coprime of $\vect{d}$ around $(\phi-1)\vect{d}$, where $\phi$ is the golden ratio, for best results.

Permutations of an $N$-element array can be represented uniquely by a each valuation of $\vect{d}=(N,N-1,\dots,2,1)$.
Iterate of all valuations of this $\vect{d}$ as shown above to get all permutations.


\end{document}

