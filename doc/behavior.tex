

\documentclass[english]{article}
\usepackage{fullpage}
\usepackage[T1]{fontenc}
\usepackage[latin9]{inputenc}
\setlength{\parskip}{\medskipamount}
\setlength{\parindent}{0pt}



\usepackage{amsmath}
\usepackage{amssymb}
\usepackage{amsthm}
\usepackage{amsfonts}
\usepackage{graphicx}
\usepackage{mathtools}
\usepackage{hyperref}

%\usepackage[g]{esvect}

\DeclarePairedDelimiter{\abs}{\lvert}{\rvert}
\DeclarePairedDelimiter{\ceil}{\lceil}{\rceil}
\DeclarePairedDelimiter{\floor}{\lfloor}{\rfloor}
\DeclarePairedDelimiter{\avect}{\langle}{\rangle}
\DeclarePairedDelimiter{\aset}{\allowbreak\lbrace}{\rbrace}

%\newcommand{\true}{\mathit{true}}
%\newcommand{\fals}{\mathit{false}}
\newcommand{\true}{\mbox{T}}
\newcommand{\fals}{\mbox{F}}
\newcommand{\Nat}{\mathbb{N}}
\newcommand{\Int}{\mathbb{Z}}
\newcommand{\defeq}{:=}
\newcommand{\transpose}{^ \top }
\newcommand{\queseq}{\overset{?}{=}}
%\newcommand{\transpose}{^ \intercal }
\newcommand{\pipe}{\,|\,}
\newcommand{\vbl}[1]{\ensuremath{\mathit{#1}}}
\newcommand{\detop}[1]{\det(#1)}
%\newcommand{\vect}[1]{\boldsymbol{\vbl{#1}}}
\newcommand{\vect}[1]{{\overrightarrow{#1}}}
\newcommand{\slfrac}[2]{\left.#1\middle/#2\right.}
%\newcommand{\vect}[1]{{\vv{#1}}}
\newcommand{\modop}[1]{\mbox{ mod }#1}
\newcommand{\mathsc}[1]{\text{\normalfont\scshape#1}}

\newcounter{exercisecnt}
\def\theexercisecnt{\arabic{exercisecnt}}
\newenvironment{exercise}
{\refstepcounter{exercisecnt}
 {\bf Exercise \theexercisecnt.}}
{}

\newcounter{exercisepartcnt}[exercisecnt]
\def\theexercisepartcnt{\theexercisecnt.\alph{exercisepartcnt}}
\newenvironment{exercisepart}
{\refstepcounter{exercisepartcnt}
 {\bf Exercise \theexercisepartcnt.}}
{}



\makeatletter
\@ifundefined{ifusesection}{%
 \newif\ifusesection %
 \usesectiontrue %
}{}
\makeatother

\ifusesection
 \newtheorem{theorem}{Theorem}[section]
\else
 \newtheorem{theorem}{Theorem}
\fi

\newtheorem{lemma}[theorem]{Lemma}
\newtheorem{corollary}[theorem]{Corollary}
\newtheorem{definition}[theorem]{Definition}
\newtheorem{example}[theorem]{Example}

\renewcommand{\labelenumii}{\arabic{enumi}.\arabic{enumii}.}
\renewcommand{\labelenumiii}{\arabic{enumi}.\arabic{enumii}.\arabic{enumiii}.}
\renewcommand{\labelenumiv}{\arabic{enumi}.\arabic{enumii}.\arabic{enumiii}.\arabic{enumiv}.}



\newcommand{\actxnto}{\,\to\,}

\begin{document}
\title{Enforcing Process Behavior}
\author{Alex Klinkhamer}
\maketitle

\section{Read-Only Variable}
If we want process $P_i$ to have a read-only variable $x$, then constrain $P_i$ by the following the following stabilizing system.
Let there be a system of two processes, $P_i$ and $P_W$, which each have a variable $x$ of the same domain size.
$P_i$ may have other variables.
Define the invariant as
\[I \equiv (P_i\oodot x = P_W\oodot x)\]
The protocol is self-stabilizing iff the following hold.
\begin{itemize}
\item $P_W$ is programmed as $P_W\oodot x \ne P_i\oodot x \actxnto P_W\oodot x\defeq P_i\oodot x$.
\item $P_i$ contains no self-loops.
\item $P_i$ never changes $P_i\oodot x$.
\end{itemize}

\section{Read-Only Variable in a System}
We can also enforce $P_i\oodot x$ is read-only by $P_i$ without constraining $P_i$ by an additional system.
Create two new processes $P_R$ and $P_W$, each with the variable $x$ of the same domain size as $P_i\oodot x$.
Let $P_W$ read $P_i\oodot x$.
Define the invariant as
\[I'\equiv (I\vee (P_i\oodot x = P_R\oodot x)) \wedge (P_i\oodot x = P_W\oodot x)\]

\section{Fixed Process Actions}
If we want to fully define actions of a process $P_i$, then constrain $P_i$ by the following stabilizing system.
Let there be a system with processes $P_C$, $P_i$, $P_W$, and $P_R$, each owning a set of variables $X$.
Define $P_i\oodot X$, $P_W\oodot X$, and $P_R\oodot X$ to hold variables of the same names and domains.
Let $P_i$ read all variables $P_C\oodot X$, noting that $P_C\oodot X$ is not similar to any of the other sets of process variables.
Let $P_W$ read $P_i\oodot X$.
Define $P_C$ and $P_R$ as read-only.
Define the invariant using $C\defeq P_C\oodot X$, $W\defeq P_W\oodot X$, and $R\defeq P_R\oodot X$.
\begin{eqnarray*}
I & \equiv & \cvbl{enable\_ck}(C \wedge P_i\oodot X) \wedge ((P_i\oodot X = W) \vee \cvbl{act}(C\wedge W \actxnto P_i\oodot X))
\\ & \vee & \cvbl{enable\_ck}(C \wedge P_i\oodot X) \wedge (P_i\oodot X = W = R)
\end{eqnarray*}
Where the \cvbl{enable\_ck} and \cvbl{act\_ck} predicates consult our desired definition for $P_i$.
That is, $\cvbl{enable\_ck}(C\wedge P_i\oodot X)$ should be true iff we want $P_i$ to transition somewhere else.
Further, $\cvbl{act\_ck}(C\wedge W \actxnto Y)$ should be true iff we want $P_i$ to have the action $C \wedge W \actxnto Y$.

\section{Token Ring of Three Bits}
Synthesizing this token ring properly is tricky because the invariant is a result of human ingenuity.
That is to say, we cannot claim a tool can synthesize a stabilizing protocol for this system if we have to provide such a clever invariant.

Consider a token ring of $1$ bit per process.
There are $N$ processes, $P_0,\dots,P_{N-1}$, each having a variable $x\in\Nat_2$.
Each $P_i$ can read $P_{i-1}$ (assume $P_{i-1\modop{N}}$), calling it $P_i\oodot \vbl{xp}$.
Processes $P_1,\dots,P_{N-1}$ are symmetric.
$P_0$ is said to have a token when $(P_0\oodot x = P_0\oodot \vbl{xp})$.
Any other other $P_i$ is said to have a token when $(P_i\oodot x \ne P_i\oodot \vbl{xp})$.

Fix $P_0$ to have the actions
\[ P_0:\; (x = \vbl{xp}) \actxnto x\defeq \neg \vbl{xp}; \]
Fix all other $P_i$ to have actions
\[ P_i:\; (x \ne \vbl{xp}) \actxnto x\defeq \vbl{xp}; \]

Let us augment this system for synthesis.

Introduce $N$ more processes $A_0,\dots,A_{N-1}$ to the system.
Each $A_i$ owns $5$ variables.
\begin{enumerate}
\item $\vbl{xp} \in \Nat_2$.
\item $\vbl{lock} \in \Nat_2$.
\item $e,t,\vbl{ready} \in \Nat_2$.
\end{enumerate}
Each $A_i$ reads $5$ variables.
\begin{enumerate}
\item $A_i\oodot\vbl{x}$ being $x$ from $P_i$.
\item $A_i\oodot\vbl{lockp}$ and $A_i\oodot\vbl{lockq}$ being \vbl{lock} from $A_{i-1}$ and $A_{i+1}$.
\item $A_i\oodot\vbl{ep}$ and $A_i\oodot\vbl{tp}$ being $e$ and $t$ from $A_{i-1}$.
\end{enumerate}

Now the important steps.
\begin{itemize}
\item Augment $P_i\oodot\vbl{xp}$ to look at $A_i\oodot\vbl{xp}$ instead of $P_{i-1}$.
\item Fix all programs for $P_i$ as defined by the non-stabilizing token ring.
\item Define the following of actions of each $A_i$.
 \begin{itemize}
 \item $A_i:\; \vbl{lock}=1 \wedge (\vbl{lockp}=1 \vee \vbl{lockq}=1) \actxnto \vbl{lock}\defeq 0;$
  \begin{itemize}
  \item Note that this also ensures $A_i$ is disabled when $\vbl{lock}=0 \wedge (\vbl{lockp}=1 \vee \vbl{lockq}=1)$.
  \end{itemize}
 \item $A_i:\; \vbl{lock}=0 \wedge \vbl{lockp}=0 \wedge \vbl{lockq}=0 \wedge \cvbl{enable\_ck}(P_i) \actxnto \vbl{lock}\defeq 1;$
  \begin{itemize}
  \item This helps define the system without overdoing it.
  \item Perhaps this would not be good in the general case.
  \end{itemize}
 \end{itemize}
\item Define the invariant as a conjunction of:
 \begin{itemize}
 \item TODO: This is too strict, closure cannot be achieved!
 \item $(\forall i\in\Nat_N: A_i\oodot\vbl{lock}=0) \implies (\exists i\in \Nat_N: \cvbl{only\_token\_ck}(i))$
  \begin{itemize}
  \item This is the original invariant on all $P_i$.
  \item Def. $\cvbl{only\_token\_ck}(i)\equiv \cvbl{token\_ck}(i) \wedge (\forall j\in\Nat_N: i=j \vee \neg \cvbl{token\_ck}(j))$
  \item Def. $\cvbl{token\_ck}(i)\equiv (i=0\wedge P_i\oodot x = P_{i-1}\oodot x) \vee (i\ne 0\wedge P_i\oodot x \ne P_{i-1}\oodot x)$
  \end{itemize}
 \item $\forall i\in\Nat_N: \cvbl{enable\_ck}(P_i) \implies A_i\oodot\vbl{lock}=1$
  \begin{itemize}
  \item Probably not needed.
  \end{itemize}
 \item $\forall i\in\Nat_N: (A_i\oodot\vbl{lock}=1) \implies (P_{i-1}\oodot x = A_i\oodot\vbl{xp})$
  \begin{itemize}
  \item This ensures that $A_i$ does not lie to $P_i$.
  \end{itemize}
 \end{itemize}
 \item Enforce that $A_i$ must have an action which is equivalent to
 \begin{eqnarray*}
 A_i: & & (\vbl{lock} = \vbl{lockp} = \vbl{lockq} = 0) \wedge \neg \cvbl{enable\_ck}(P_i) \wedge (\vbl{xp} \ne P_{i-1}\oodot x)
 \\& \actxnto & \vbl{xp} \defeq P_{i-1}\oodot x;
 \\& & \vbl{lock} \defeq 1 \mbox{ or remains } 0;
 \end{eqnarray*}
  \begin{itemize}
  \item TODO
  \item Do this by constraining $A_i$ by another system containing only $P_i$, $P_{i-1}$, $A_i$, and a new process $D$ which has all the variables of $A_{i-1}$ which are read by $P_{i-1}$ and $A_i$.
  \item Invariant $I \equiv \neg \cvbl{enable\_ck}(P_{i-1}) \wedge (A_i\oodot\vbl{xp} = P_{i-1}\oodot x)$
  \end{itemize}
\end{itemize}

\end{document}

