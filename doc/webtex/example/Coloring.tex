
\title{Coloring}
%\author{}
\date{}

\begin{document}

\tableofcontents

\section{3-Coloring on a Ring}
\label{sec:ColorRing}

\quicksec{ColorRing}
(\href{\examplespec/ColorRing.prot}{spec},
\href{\examplesett/ColorRing.args}{args},
(\href{\examplespec/ColorRing.prot}{synt},
\href{\examplesoln/ColorRing.prot}{soln})

The \ilfile{examplespec/ColorRing.prot} file specifies a bidirectional ring topology where each process wishes to obtain a different value than its neighbors.
Each process has $3$ colors to choose from.
This is a very simple protocol and is useful for instruction.

\codeinputlisting{../../../examplespec/ColorRing.prot}
The invariant can also be specified outside of the process definition as:
\begin{code}
(future & silent)
  (forall i <- Nat % N : c[i] != c[i+1]);
\end{code}
outside of the process definition.

The commands in this tutorial should be executed from within the top-level project directory on a Linux system.
First make a directory \ilfile{tmp/} to store some files.
\begin{code}
mkdir tmp
\end{code}

\subsection{Synthesis}
Synthesize a 3-coloring protocol that works on rings of size $3$, $4$, and $5$.
\begin{code}
./bin/protocon -x examplespec/ColorRing.prot -def N 3 -param N 4 -param N 5 -o tmp/color.prot
\end{code}

\subsection{Verification}
Check if the protocol is self-stabilizing rings of size $7$. It should be fine!
\begin{code}
./bin/protocon -verify -x tmp/color.prot -def N 7
\end{code}

\subsection{Model Checking with Spin}
We can also verify this protocol's correctness in the \href{http://spinroot.com}{Spin model checker} (here's a \href{../../tut/spin.html}{quick setup guide}).
First output a model in the Promela language for a ring of $6$ processes.
\begin{code}
./bin/protocon -nop -x tmp/color.prot -def N 6 -o-promela tmp/color.pml
\end{code}
Now open the model in iSpin.
\begin{code}
cd tmp
ispin color.pml
\end{code}
In the \ilkey{Edit/View} tab, add the following lines to the end of the file.
This defines a predicate \ttvbl{Legit} which evaluates to true in all legitimate states.
There are also two temporal properties which define closure and convergence.
Closure ensures that once the ring has a valid coloring, no process will change its state in a way which violates that property.
Convergence ensures that from any initial state, the ring will eventually reach a valid coloring.
\begin{code}
#define Legit \
  (c[0] != c[1] && c[1] != c[2] && \
   c[2] != c[3] && c[3] != c[4] && \
   c[4] != c[5] && c[5] != c[0])
ltl Closure { [] (Legit -> [] Legit) }
ltl Convergence { [] <> Legit }
\end{code}
Perform a \ilkey{Syntax Check}; there should be nothing to report.
Now switch to the \ilkey{Verification} tab so we can verify that closure and convergence hold.
Under \ilkey{Liveness}, ensure that the \ilkey{acceptance cycles} bubble is filled.
Next, under \ilkey{Never Claims}, click \ilkey{use claim} and type \ilkey{Closure} for the \ilkey{claim name}.
Verify closure holds by clicking \ilkey{Run}.
Now verify that covergence holds.
\textit{Can you make a change to the 3-coloring protocol that breaks closure? How about convergence?}

\subsection{Parallel Search}
When a problem instance is difficult, it may help to perform multiple searches at the same time using the \ilflag{-parallel} flag.
Since multiple searches are occurring simultaneously, the search progress output is hidden.
This is remedied with the \ilflag{-o-log} flag to create log files \ilfile{search.log.}\ilsym{tid} for each thread of index \ilsym{tid}.
\begin{code}
./bin/protocon -parallel -x examplespec/ColorRing.prot -def N 3 -param N 4 -param N 5 -o tmp/color.prot -o-log tmp/color.log
\end{code}
If these log files are not desired, simply do not give the flag.

To control the number of threads used at runtime, simply give a number after the \ilflag{-parallel} flag.
For example, imagine we want to use exactly $2$ threads.

\begin{code}
./bin/protocon -parallel 2 -x examplespec/ColorRing.prot -def N 5
\end{code}

\section{Distributed 3-Coloring on a Ring}
\label{sec:ColorRingDistrib}

\quicksec{ColorRingDistrib}
(\href{\examplespec/ColorRingDistrib.prot}{spec},
\href{\examplesett/ColorRingDistrib.args}{args},
\href{\examplesoln/ColorRingDistrib.prot}{soln})

\section{Distance-2 5-Coloring on a Ring}
\label{sec:ColorRingLocal}

\quicksec{ColorRingLocal}
(\href{\examplespec/ColorRingLocal.prot}{spec},
\href{\examplesett/ColorRingLocal.args}{args},
\href{\examplesoln/ColorRingLocal.prot}{soln})

\end{document}

