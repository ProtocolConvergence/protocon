
\title{One-Bit Maximal Matching on a Ring}
%\author{}
\date{}

\begin{document}

\href{http://en.wikipedia.org/wiki/Matching_(graph_theory)}{Matching} is well-known problem from graph theory.
A matching is a set of edges which do not share any common vertices.
For a matching to be \textit{maximal}, it must be impossible to add another edge to the set without breaking the matching property.

\tableofcontents

\section{1-Bit Maximal Matching on a Ring}
\label{sec:MatchRing}

\quicksec{MatchRing}
(\href{\examplespec/MatchRing.prot}{spec},
\href{\examplesett/MatchRing.args}{args},
\href{\examplesynt/MatchRing.prot}{synt}
\href{\examplesoln/MatchRingOneBit.prot}{soln})

In the specification, the $e$ variables denote whether an edge is in the matching.
The invariant specified as a maximal matching, which can be restated as the following two conditions for the special case of a ring:
\begin{enumerate}
\item No two adjacent edges can be selected.
\item At least one of every three consecutive edges must be selected.
\end{enumerate}

\quicksec{Synthesis}
Processes cannot realistically write to edge variables, therefore the $e$ variables are marked as \ilcode{shadow} and their values must be derived from $x$ values owned by processes.
For an instructive look at how this works see \href{#sec:MatchRingOneBit}{the next section}, which derives the same protocol from a slightly different way of specifying the maximal matching property.

\quicksec{Stabilization Proof}
It is fairly easy to show that the 1-bit matching protocol is stabilizing.
First we will show that all executions terminate.
Then we will show that all silent states belong to the invariant.
From the \href{\examplesoln/MatchRingOneBit.prot}{protocol}, we see that each $P[i]$ has the following actions:
\begin{code}
(              x[i]==1 && x[i+1]==1 --> x[i]:=0; )
( x[i-1]==0 && x[i]==0 && x[i+1]==0 --> x[i]:=1; )
\end{code}

We can analyze the actions to see that the protocol is livelock-free.
The first action of $P[i]$ removes cases of $2$ consecutive $1$ values by changing the left value to $0$.
This may enable the second action of $P[i-1]$.
The second action of $P[i]$ removes cases of $3$ consecutive $0$ values by changing the middle value to $1$.
If $P[i]$ executes its section action neither it or its neighbors is enabled!
Therefore, actions may propagate by changing consecutive $1$s to $0$s, and some of the resulting $0$s may toggle back to $1$s, but the $1$s will not be consecutive.

Clearly the silent states are those where no $2$ consecutive $1$s exist and no $3$ consecutive $0$s exist.
That means a $1$ must occur at least every $3$ values and will be followed by a $0$.
We can therefore interpret a $1$ value followed by a $0$ value to mean that the edge between the two values is selected.
These edges are not consecutive and at least on of every $3$ will be selected, therefore it is a maximal matching.
\begin{code}
forall i <- Nat % N :
   x[i-1]==1 && x[i]==0               // P[i] matched with P[i-1]
|| x[i-1]==0 && x[i]==0 && x[i+1]==1  // P[i] is not matched
||              x[i]==1 && x[i+1]==0  // P[i] matched with P[i+1]
\end{code}

\section{3-State Maximal Matching on a Ring}
\label{sec:MatchRingThreeState}

\quicksec{MatchRingThreeState}
(\href{\examplespec/MatchRingThreeState.prot}{spec},
\href{\examplesett/MatchRingThreeState.args}{args},
\href{\examplesoln/MatchRingThreeState.prot}{soln})

Matching can also be reasoned about in terms of processes.
Allow each process $P[i]$ in a ring to point to $P[i-1]$, itself, or $P[i+1]$.
Let these directions be denoted by having $P[i]$'s variable $m[i]$ have a value of $L$, $S$, and $R$ respectively.
The processes form a maximal matching when they point to each other or themselves, but no two neighboring processes can both point to themselves.
\begin{code}
forall i <- Nat % N :
   m[i-1]==R && m[i]==L               // P[i] pointing to P[i-1] and P[i-1] pointing back
|| m[i-1]==L && m[i]==S && m[i+1]==R  // P[i] pointing to itself and neighbors pointing away
||              m[i]==R && m[i+1]==L  // P[i] pointing to P[i+1] and P[i+1] pointing back
\end{code}

One stabilizing protocol has the actions:
\begin{code}
( m[i-1]==2 && m[i]!=0 && m[i+1]!=0 --> m[i]:=0; )
( m[i-1]!=2 && m[i]!=1 && m[i+1]==2 --> m[i]:=1; )
( m[i-1]!=2 && m[i]!=2 && m[i+1]!=2 --> m[i]:=2; )
\end{code}

\subsection{Deriving 1-Bit Protocol from 3-State Protocol}
\label{sec:MatchRingOneBit}

\quicksec{MatchRingOneBit}
(\href{\examplespec/MatchRingOneBit.prot}{spec},
\href{\examplesynt/MatchRingOneBit.prot}{synt},
\href{\examplesoln/MatchRingOneBit.prot}{soln})

This section explains shadow/puppet synthesis as a special case of superposition.
In the \href{#sec:MatchRing}{previous section}, we saw that a protocol could achieve a matching using only 1 bit per process.
How could this be derived?
From the above 3-state protocol, it seems that each process needs to be able to point in $3$ directions.

Give each process $P[i]$ a binary $x[i]$ variable to perform the protocol along with a ternary $m[i]$ variable used to specify the invariant.
Furthermore, $P[i]$ is given read access to $x[i-1]$ and $x[i+1]$, but it cannot read $m[i-1]$ or $m[i+1]$.

We use the previous section's invariant on the underlying $m$ variables.
Since processes only know their own $m$ values, the protocol is forced to use $x$ values to negotiate appropriate $m$ values.
Our invariant style (the \ilcode{((future & silent) % puppet)} style) allows closure to be violated for some (but not all) valuations of $x$ variables.
\codeinputlisting{../../../examplespec/MatchRingOneBit.prot}

From synthesis, \href{\examplesynt/MatchRingOneBit.prot}{one of the protocols} we get is the following.
\begin{code}
( x[i-1]==1 && x[i]==1 && x[i+1]==1 --> x[i]:=0; m[i]:=L; )
( x[i-1]==0 && x[i]==1 && x[i+1]==1 --> x[i]:=0; m[i]:=S; )
( x[i-1]==0 && x[i]==0 && x[i+1]==0 --> x[i]:=1; m[i]:=R; )

( x[i-1]==1 && x[i]==0              && m[i]!=L --> m[i]:=L; )
( x[i-1]==0 && x[i]==0 && x[i+1]==1 && m[i]!=S --> m[i]:=S; )
( x[i-1]==0 && x[i]==1 && x[i+1]==0 && m[i]!=R --> m[i]:=R; )
\end{code}

From here, we can create the 1-bit matching protocol on the $x[i]$ variables without the $m[i]$s.
The first three actions of the synthesized protocol change $x[i]$ and are therefore used as actions in our 1-bit matching protocol, discarding changes to $m[i]$.
\begin{code}
( x[i-1]==1 && x[i]==1 && x[i+1]==1 --> x[i]:=0; )
( x[i-1]==0 && x[i]==1 && x[i+1]==1 --> x[i]:=0; )
( x[i-1]==0 && x[i]==0 && x[i+1]==0 --> x[i]:=1; )
\end{code}

The invariant is all states where the $x[i]$ values don't change (see the last three actions above).
\begin{code}
forall i <- Nat % N :
   x[i-1]==1 && x[i]==0               // P[i] pointing to P[i-1] and P[i-1] pointing back
|| x[i-1]==0 && x[i]==0 && x[i+1]==1  // P[i] pointing to itself and neighbors pointing away
|| x[i-1]==0 && x[i]==1 && x[i+1]==0  // P[i] pointing to P[i+1] and P[i+1] pointing back
\end{code}
Each of these cases in the disjunction corresponds to $P[i]$ pointing to $P[i-1]$, itself, and $P[i+1]$ respectively.
We know this by looking at how $m[i]$ is changed to be \ilcode{m[i]:=L}, \ilcode{m[i]:=S}, and \ilcode{m[i]:=R} in the synthesized protocol.

Note that the third case in the disjunction can be simplified from \ilcode{x[i-1]==0 && x[i]==1 && x[i+1]==0} to \ilcode{x[i]==1 && x[i+1]==0} since if the formula holds for $P[i]$ and the system is in the invariant, then the first or second cases in the disjunction must hold for $P[i-1]$ (hence, $x[i-1]=0$).

Putting this all together, we get:
\codeinputlisting{../../../examplesoln/MatchRingOneBit.prot}

\subsection{Using Shadow/Puppet Variables}

Shadow variables are variables that cannot be used in the guard of any actions.
Therefore, to reliably obtain a protocol that is free of $m$ variables, we must mark them as shadow variables by replacing \ilcode{direct} with \ilcode{shadow} in the \href{\examplespec/MatchRingOneBit.prot}{specification}.
This is essentially what is done in the \href{#sec:MatchRing}{previous section}, but a more convenient invariant is used.

\section{Segmented Ring}
\label{sec:SegmentRing}

\quicksec{SegmentRing}
(\href{\examplespec/SegmentRing.prot}{spec},
\href{\examplesett/SegmentRing.args}{args},
\href{\examplesynt/SegmentRing.prot}{synt})

This is a problem similar to matching where a ring is segmented into chains.

\end{document}

