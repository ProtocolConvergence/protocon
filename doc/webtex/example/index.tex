
\title{Example List}
%\author{}
\date{}

\newcommand{\LinkText}{[info]}

\begin{document}

Example files are given in three directories: (1) \href{\examplespec}{examplespec} contains problem specifications, (2) \href{\examplesoln}{examplesoln} contains solutions, and (3) \href{\examplesynt}{examplesynt} contains solutions that retain artifacts from the specification.
For each file in \href{\examplesynt}{examplesynt}, there is usually a corresponding file in \href{\examplesoln}{examplesoln} that is more presentable and simpler to verify.

\tableofcontents

\section{Sum-Not-2}

Aly Farahat's \href{http://digitalcommons.mtu.edu/etds/178}{dissertation} includes a simple yet nontrivial unidirectional ring protocol.
It has been extremely helpful for reasoning about the nature of \href{http://dx.doi.org/10.1007/978-3-319-03089-0\_12}{livelocks in unidirectional rings} and has a simple enough transition system to actually show in a presentation.
\begin{itemize}
\item SumNotTwo (\href{\examplespec/SumNotTwo.prot}{spec})
\item SumNotTarget (\href{\examplespec/SumNotTarget.prot}{spec}, \href{\examplesoln/SumNotTarget.prot}{soln})
- The general case sum-not-$(l-1)$ protocol given by Farahat.
\end{itemize}

\section{Coloring}

\href{http://en.wikipedia.org/wiki/Graph_coloring}{Graph coloring} is a well-known problem with many applications.
Each node in the graph is assigned a color.
For this assignment to be called a \textit{coloring}, each node must have a different color than the nodes adjacent to it.
In a computer network, a coloring applies to processes that communicate directly with each other rather than nodes connected by edges.
\begin{itemize}
\item ColorRing \href{Coloring.html#sec:ColorRing}{\LinkText}
- 3-coloring on a ring.
\item ColorRingDizzy (\href{\examplespec/ColorRingDizzy.prot}{spec}, \href{\examplesoln/ColorRingDizzy.prot}{soln})
- Unoriented ring.
\item ColorUniRing (\href{\examplespec/ColorUniRing.prot}{spec}, \href{\examplesoln/ColorUniRing.prot}{soln})
- Randomized 3-coloring on a unidirectional ring.
\item ColorRingLocal \href{Coloring.html#sec:ColorRingLocal}{\LinkText}
- Randomized distance-2 coloring on a unidirectional ring using $5$ colors.
Neither a process or its neighbors can have the same color as another neighbor.
\item ColorRingDistrib \href{Coloring.html#sec:ColorRingDistrib}{\LinkText}
- Randomized 3-coloring on a ring where processes have communication delay.
\item ColorChain (\href{\examplespec/ColorChain.prot}{spec})
- 2-coloring on a chain.
\item ColorTree (\href{\examplespec/ColorTree.prot}{spec}, \href{\examplesoln/ColorTree.prot}{soln})
- Tree.
\item ColorTorus (\href{\examplespec/ColorTorus.prot}{spec})
- Torus.
\item ColorMobius (\href{\examplespec/ColorMobius.prot}{spec})
- Mobius ladder.
\item ColorKautz (\href{\examplespec/ColorKautz.prot}{spec})
- 4-coloring on generalized Kautz graph of degree 2.
We can find a protocol that stabilizes for up to $13$ processes, but not $14$.
Tweaking the topology by
\href{\examplespec/ColorKautzReverse.prot}{reversing edges}
or
\href{\examplespec/ColorKautzDizzy.prot}{removing orientation}
(even with \href{\examplespec/ColorKautzBi.prot}{bidirectional edges})
gives definite impossibility results at around $8$ processes.
\end{itemize}

\section{Maximal Matching}

\href{http://en.wikipedia.org/wiki/Matching_(graph_theory)}{Matching} is well-known problem from graph theory.
A matching is a set of edges that do not share any common vertices.
For a matching to be \textit{maximal}, it must be impossible to add another edge to the set without breaking the matching property.
\begin{itemize}
\item MatchRing \href{Matching.html#sec:MatchRing}{\LinkText}
- Natural specification for matching using the edges in the ring.
This gives the 1-bit solution.
\item MatchRingThreeState \href{Matching.html#sec:MatchRingThreeState}{\LinkText}
- Maximal matching on a ring where processes point in certain directions.
\item MatchRingOneBit \href{Matching.html#sec:MatchRingOneBit}{\LinkText}
- Using the 3-state specification, find a matching using 1 bit per process.
\item MatchRingDizzy (\href{\examplespec/MatchRingDizzy.prot}{spec}, \href{\examplesett/MatchRingDizzy.args}{args})
- Maximal matching on an unoriented ring.
\item SegmentRing \href{Matching.html#sec:SegmentRing}{\LinkText}
- A problem similar to matching where a ring is segmented into chains.
\end{itemize}

\section{Sorting on Chains and Rings}

\begin{itemize}
\item SortChain (\href{\examplespec/SortChain.prot}{spec}, \href{\examplesoln/SortChain.prot}{soln})
- Sorting a chain of values.
Easy to synthesize, but solution is manually simplified.
\item SortRing (\href{\examplespec/SortRing.prot}{spec}, \href{\examplesoln/SortRing.prot}{soln})
- Sorting a ring of values using a unique zero value to mark the beginning of the sequence.
Easy to synthesize, but solution is manually simplified.
\end{itemize}

\section{Orientation}

\begin{itemize}
\item OrientDaisy \href{Orientation.html#sec:OrientDaisy}{\LinkText}
- Silent orientation for daisy chains (either ring and chain), verified for $2$ to $15$ processes.
The version in \href{\examplesoln}{examplesoln} behaves similarly, is simpler, but is slightly less optimal.
\item OrientRing \href{Orientation.html#sec:OrientRing}{\LinkText}
- Manually designed silent ring orientation, verified for $2$ to $26$ processes.
Also works under synchronous scheduler.
\item OrientRingOdd \href{Orientation.html#sec:OrientRingOdd}{\LinkText}
- From \href{http://dx.doi.org/10.1007/BFb0020439}{Hoepman}.
\item OrientRingViaToken (\href{\examplespec/OrientRingViaToken.prot}{spec}, \href{\examplesoln/OrientRingViaToken.prot}{soln})
- From \href{http://dx.doi.org/10.1006/inco.1993.1029}{Israeli and Jalfon}.
\end{itemize}

\section{Token Passing}

\begin{itemize}
\item TokenRingFiveState \href{TokenPassing.html#sec:TokenRingFiveState}{\LinkText}
- 5-state token ring whose behavior is specified with shadow variables.
\item TokenRingSixState \href{TokenPassing.html#sec:TokenRingSixState}{\LinkText}
- 6-state token ring specified with variable superposition.
\item TokenRingThreeBit \href{TokenPassing.html#sec:TokenRingThreeBit}{\LinkText}
- 3-bit token ring from \href{http://dx.doi.org/10.1006/jpdc.1996.0066}{Gouda and Haddix}.
\item TokenRingFourState (\href{\examplesynt/TokenRingFourState.prot}{synt})
- 4-state token ring specified with variable superposition.
This version is only stabilizing for $2$ to $7$ processes.
It operates much like the 3-bit token ring.
\item TokenRingDijkstra \href{TokenPassing.html#sec:TokenRingDijkstra}{\LinkText}
- Dijkstra's stabilizing token ring.
\item TokenChainThreeState \href{TokenPassing.html#sec:TokenChainThreeState}{\LinkText}
- 3-state token passing on a chain topology.
\item TokenChainDijkstra \href{TokenPassing.html#sec:TokenChainDijkstra}{\LinkText}
- Dijkstra's 4-state token passing on a chain topology.
\item TokenRingThreeState \href{TokenPassing.html#sec:TokenRingThreeState}{\LinkText}
- 3-state token passing on a bidirectional ring.
One solution is from Dijkstra, the other is from \href{http://citeseerx.ist.psu.edu/viewdoc/summary?doi=10.1.1.153.6017}{Chernoy, Shalom, and Zaks}.
\item TokenRingOdd (\href{\examplespec/TokenRingOdd.prot}{spec}, \href{\examplesoln/TokenRingOdd.prot}{soln})
- Randomized token passing protocol on odd-sized rings.
\end{itemize}

%\section{Leader Election}
%\section{Reduction from 3-SAT}

\section{Other}

\begin{itemize}
\item ByzantineGenerals (\href{\examplespec/ByzantineGenerals.prot}{spec}, \href{\examplesoln/ByzantineGenerals.prot}{soln})
- The Byzantine generals problem formulated as an instance of self-stabilization.
\item DiningCrypto (\href{\examplespec/DiningCrypto.prot}{spec}, \href{\examplesoln/DiningCrypto.prot}{soln})
- The dining cryptographers problem as an instance of self-stabilization, where the initial state is assumed to randomize the coins.
We can't model anonymity, only determination of whether a cryptographer or the NSA pays for dinner.
\item DiningPhilo (\href{\examplespec/DiningPhilo.prot}{spec}, \href{\examplesoln/DiningPhilo.prot}{soln})
- The dining philosophers problem. This version assumes a coloring in order to break symmetry.
\item DiningPhiloRand (\href{\examplespec/DiningPhiloRand.prot}{spec}, \href{\examplesoln/DiningPhiloRand.prot}{soln})
- The dining philosophers problem. This version uses randomization to break symmetry.
\item LeaderRingHuang (\href{\examplespec/LeaderRingHuang.prot}{spec}, \href{\examplesoln/LeaderRingHuang.prot}{soln})
- Leader election protocol on prime-sized rings from \href{http://dx.doi.org/10.1145/169683.174161}{Huang}.
\item Sat (\href{\examplespec/Sat.prot}{spec})
- Example reduction from 3-SAT to adding stabilization from our \href{http://dx.doi.org/10.1007/978-3-642-40213-5_2}{paper showing NP-completeness} of adding convergence.
\item StopAndWait (\href{\examplespec/StopAndWait.prot}{spec}, \href{\examplesoln/StopAndWait.prot}{soln})
- The Stop-and-Wait protocol, otherwise known as the Alternating Bit protocol when the sequence number is binary.
\end{itemize}

\end{document}

