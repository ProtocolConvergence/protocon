
\title{Coloring}
%\author{}
\date{}

\begin{document}

The \ilfile{inst/ColorRing.protocon} file specifies a bidirectional ring topology where each process wishes to obtain a different value than its neighbors.

This is a very simple protocol, and is useful for instruction.
Use the \ilflag{-simple} flag to see non-random run without OpenMP or restarts.
Use the \ilflag{-def} flag to set the constant \illit{N} equal to $5$, corresponding to $5$ processes in the ring.
\begin{code}
protocon -simple -x inst/ColorRing.protocon -o found.protocon -def N 5
\end{code}

If there are more cores available, run
\begin{code}
protocon -x inst/ColorRing.protocon -o found.protocon -def N 5 -o-log search.log
\end{code}
We use the \ilflag{-o-log} flag to create log files \ilfile{search.log.}\ilsym{tid} for each thread of index \ilsym{tid}.
If these log files are not desired, simply do not give the flag.

To use the default search method (using random choinces and restarts) with only one thread (and see the output), set the \ilname{$OMP_NUM_THREADS} environment variable to $1$.

In the \ilname{bash} shell:
\begin{code}
export OMP_NUM_THREADS=1
protocon -x inst/ColorRing.protocon -def N 5
unset OMP_NUM_THREADS
\end{code}

This can of course be accomplished on one line:
\begin{code}
OMP_NUM_THREADS=1 protocon -x inst/ColorRing.protocon -def N 5
\end{code}
The equivalent \ilname{csh} or \ilname{tcsh} shell commands are:
\begin{code}
setenv OMP_NUM_THREADS 1
protocon -x inst/ColorRing.protocon -def N 5
unsetenv
\end{code}

\end{document}

