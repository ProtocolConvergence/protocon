

%preamble

\documentclass[english]{article}
\usepackage{fullpage}
\usepackage[T1]{fontenc}
\usepackage[latin9]{inputenc}
\setlength{\parskip}{\medskipamount}
\setlength{\parindent}{0pt}

\usepackage{amsmath}
\usepackage{amssymb}
\usepackage{amsthm}
\usepackage{amsfonts}
\usepackage{graphicx}
\usepackage[all,arc,curve,frame]{xy}

\newcommand{\Nat}{\mathbb{N}}
\newcommand{\defeq}{:=}
\newcommand{\vect}[1]{\boldsymbol{#1}}

\renewcommand{\labelenumii}{\arabic{enumi}.\arabic{enumii}.}
\renewcommand{\labelenumiii}{\arabic{enumi}.\arabic{enumii}.\arabic{enumiii}.}
\renewcommand{\labelenumiv}{\arabic{enumi}.\arabic{enumii}.\arabic{enumiii}.\arabic{enumiv}.}



\begin{document}
\title{Reduce AddConvergence to SAT}
\author{Alex Klinkhamer}
\maketitle

\section{Notation}
States will be called $s_i$.
Local states of process $r$ be called $s_i^r$.
A transition from $s_i$ to $s_j$ is called $t_{ij}$.
An action (or rule) from $s_i^r$ to $s_j^r$ is called $g_{ij}^r$.
For brevity, actions can be called $g_i$, with $i$ enumerating all potential actions.
The reachability from $s_i$ to $s_j$ shall be $p_{ij}$.

When used in a boolean formula, assume these symbols evaluate to their corresponding propositional variables.

\section{Reduction}
Transition from every illegitimate state.
$$\forall i:\exists j: t_{ij}$$

Preserve protocol in legitimate states.
Assure each $t_{ij}$ in the original protocol ($s_i$ and $s_j$ are legitimate) exists. That is, $$\forall t_{ij}: t_{ij}$$

Actions and transitions correspond.
$$\forall g_{ij}^r:\forall t_{kl}\in g_{ij}^r: \neg g_{ij}^r \vee t_{kl}$$
$$\forall t_{ij}:\exists g_{kl}^r: \left(t_{ij}\in g_{kl}^r\right) \wedge\left(\neg t_{ij}\vee g_{kl}^r\right)$$

At most $1$ action can be used for each local state.
$$\forall a_{ij}^r, a_{ik}^r: \neg a_{ij}^r \vee \neg a_{ik}^r$$

No cycles outside the illegitimate states.
This one is a pain.
$$\forall p_{ik},t_{kj}: q_{ikj}\iff p_{ik} \wedge t_{kj}$$
$$\forall p_{ij}: p_{ij}\iff t_{ij}\vee\exists k: q_{ikj}$$
$$\forall s_i\notin \overline{I}: \neg p_{ii}$$

\end{document}

